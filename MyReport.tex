% !TEX TS-program = pdflatex
% !TEX encoding = UTF-8 Unicode

% This is a simple template for a LaTeX document using the "article" class.
% See "book", "report", "letter" for other types of document.

\documentclass[11pt]{report} % use larger type; default would be 10pt

\usepackage[utf8]{inputenc} % set input encoding (not needed with XeLaTeX)

%%% Examples of Article customizations
% These packages are optional, depending whether you want the features they provide.
% See the LaTeX Companion or other references for full information.

%%% PAGE DIMENSIONS
\usepackage{geometry} % to change the page dimensions
\geometry{a4paper} % or letterpaper (US) or a5paper or....
% \geometry{margin=2in} % for example, change the margins to 2 inches all round
% \geometry{landscape} % set up the page for landscape
%   read geometry.pdf for detailed page layout information

\usepackage{graphicx} % support the \includegraphics command and options

% \usepackage[parfill]{parskip} % Activate to begin paragraphs with an empty line rather than an indent

%%% PACKAGES
\usepackage{booktabs} % for much better looking tables
\usepackage{array} % for better arrays (eg matrices) in maths
\usepackage{paralist} % very flexible & customisable lists (eg. enumerate/itemize, etc.)
\usepackage{verbatim} % adds environment for commenting out blocks of text & for better verbatim
\usepackage{subfig} % make it possible to include more than one captioned figure/table in a single float
% These packages are all incorporated in the memoir class to one degree or another...

%%% HEADERS & FOOTERS
\usepackage{fancyhdr} % This should be set AFTER setting up the page geometry
\pagestyle{fancy} % options: empty , plain , fancy
\renewcommand{\headrulewidth}{0pt} % customise the layout...
\lhead{}\chead{}\rhead{}
\lfoot{}\cfoot{\thepage}\rfoot{}

%%% SECTION TITLE APPEARANCE
\usepackage{sectsty}
\allsectionsfont{\sffamily\mdseries\upshape} % (See the fntguide.pdf for font help)
% (This matches ConTeXt defaults)

%%% ToC (table of contents) APPEARANCE
\usepackage[nottoc,notlof,notlot]{tocbibind} % Put the bibliography in the ToC
\usepackage[titles,subfigure]{tocloft} % Alter the style of the Table of Contents
\renewcommand{\cftsecfont}{\rmfamily\mdseries\upshape}
\renewcommand{\cftsecpagefont}{\rmfamily\mdseries\upshape} % No bold!

%%%Chapter style
\usepackage[explicit]{titlesec}
\usepackage{lmodern}
\usepackage{lipsum}

\newlength\chapnumb
\setlength\chapnumb{4cm}

\titleformat{\chapter}[block]
{\normalfont\sffamily}{}{0pt}
{\parbox[b]{\chapnumb}{%
   \fontsize{120}{110}\selectfont\thechapter}%
  \parbox[b]{\dimexpr\textwidth-\chapnumb\relax}{%
    \raggedleft%
    \hfill{\LARGE#1}\\
    \rule{\dimexpr\textwidth-\chapnumb\relax}{0.4pt}}}
\titleformat{name=\chapter,numberless}[block]
{\normalfont\sffamily}{}{0pt}
{\parbox[b]{\chapnumb}{%
   \mbox{}}%
  \parbox[b]{\dimexpr\textwidth-\chapnumb\relax}{%
    \raggedleft%
    \hfill{\LARGE#1}\\
    \rule{\dimexpr\textwidth-\chapnumb\relax}{0.4pt}}}
    
\setlength{\parindent}{2em}
\setlength{\parskip}{1em}
%%% END Article customizations

%%% The "real" document content comes below...

\title{Brief Article}
\author{NGUYEN Quang Anh}

%\date{} % Activate to display a given date or no date (if empty),
         % otherwise the current date is printed 

\begin{document}
\maketitle
\newpage
\begin{center}
{\huge Abstract}
\end{center}

Mobile development is one of the most popular industries nowaday. Each year there are several hundreds millions of smartphones which were sold, in which Android devices and iOS devices hold the most part, around 98\% of the market. I decided to learn about Android development, since it used Java language, the most familiar programming language to me. For that reason I applied into a branch of Rainmaker Labs in Vietnam, a mobile development outsourcing company.

%%%chapter 1
\chapter{Introduction}
\section{Rainmaler Labs}

Rainmaker Labs is the singaporean outsourcing company, specialized in mobile development, currently placed first among the competitors in Singapore and Asia Pacific. 

Rainmaker Labs in Vietnam is the new branch which was established in Feb 2015. Though it's new, but thanks to the politics, cultures and remuneration policy, the offshore branch was able to recruit many talented and experienced individuals, some of them even hold high position in their previous jobs.

\section{Projects}

Although I was new to Android at that time, infact, I had never touch Android before, but I still managed to convince my seniors to let me participate in company's projects. The first project was the small project, named \textbf{Pedro}, which served at the \textbf{Charle \& Keith Fasionable Awards 2015}. In this project I was able to study about the basic of Android development and some advanced technic.

The second one is a big project. This project is the digital version of Singapore government project, \textbf{Electronic Tourist Refund Scheme} aka \textbf{eTRS}. In this report I will concentrate on the technics, and knownledge which I was able to acquired. 


%%%chapter 2
\chapter{Software Development Method}

In Rainmaker Labs, we use \textbf{SCRUM} as our software deployment method. With this method, we could deploy software to the customer as fast as possible, and be closest to customer's expectation. There are some important points such as :

\begin{itemize}
	\item break down the project into small tasks, as small as possible.
	\item estimate the difficulties of each task and give points to them, then each member of team will select their task, base on their capability, such that all the team members have the same amount of points.
	\item during the development process, whenever a member encounter a problem which he couldn't solve, they could always ask for help. Team leader, or someone else who did solve the problem before could give hints to help him resolve his issue.
	\item everyday before starting working, team will have a stand-up meeting for about 10 minutes. Scrum master would hear team member reporting about their processes the previous day, and their planning for today's works. 
	\item after each sprint (about a week), team will demonstrate what they have done to the client, allowing them to follow the development process, and correct the features to their ideal.
	\item after some sprints, when the product is about to finish, the tester team will join and dug in for bug. The development team now will start to fix the reported bug, along with working on their tasks.
\end{itemize}

%%%chapter 3
\chapter{How I understand about Android development}

Although Android use Java as its programming language, writing an Android application is more complicated than writing a Java one.

\section{Activity and Fragment}

\subsection{Activity}
One of the most important component in Android is \textit{Activity}. An \textit{Activity} represent a screen in an application, and an instance of this kind of class is not initialized by user, infact, it is created by the Android's system, by injection via a file \textit{manifest.xml}. For that reason, transfer data between \textit{Activities} can't be done by the usual ways. Using static variable is no good neither, because each \textit{Activity} has their own life circle, when its life circle ends, the system will kill the \textit{Activity}, and garbage collector will free the memory zone which was hold by this \textit{Activity}, that makes all the static variables reset. 
\subsection{Fragment}
A \textit{Fragment} is a portion of an \textit{Activity}, it could be added or removed when an \textit{Activity} is active. A \textit{Fragment} also has its own life circle.

\end{document}