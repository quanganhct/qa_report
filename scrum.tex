\chapter{Software Development Method}

In Rainmaker Labs, we use \textbf{SCRUM} as our software deployment method. With this method, we could deploy software to the customer as fast as possible, and be closest to customer's expectation. There are some important points in SCRUM such as :

\begin{itemize}
	\item break down the project into small tasks, as small as possible.
	\item estimate the difficulties of each task and give points to them, then each member of team will select their task, base on their capability, such that all the team members have the same amount of points.
	\item during the development process, whenever a member encounter a problem which he couldn't solve, they could always ask for help. Team leader, or someone else who did solve the problem before could give hints to help him resolve his issue.
	\item everyday before starting working, team will have a stand-up meeting for about 10 minutes. Scrum master would hear team member reporting about their processes the previous day, and their planning for today's works. 
	\item after each sprint (about a week), team will demonstrate what they have done to the client, allowing them to follow the development process, and correct the features to their ideal.
	\item after some sprints, when the product is about to finish, the tester team will join and dug in for bug. The development team now will start to fix the reported bug, along with working on their tasks.
\end{itemize}

Sadly in our company, SCRUM is not deployed as 100\% as it means to be. The estimation of tasks is not executed by the whole team. In fact, it was the 2 most experience in our team that broken down and estimated every tasks, turn out that the estimation is not done objectively. Some tasks were over-estimated, some were under-estimated. And the most critical point is that both Android and iOS team have the same point for the same task, which is not true. Since there're tasks that could be done easily in Android while in iOS they have to struggle to make it right, and reverse. 

One more drawback, is that actually in our company, point is equal to hour. So a 4 points task is given 4 hours to complete. This is also not true, since point should be given based on the complexity of the task itself. 

Anyway, it was interested to experience method SCRUM myself, though it was not 100\% it supposed to be. 